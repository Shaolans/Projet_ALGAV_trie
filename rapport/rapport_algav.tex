\documentclass[a4paper,12pt]{report}
%\documentclass[a4paper,10pt]{scrartcl}

\usepackage[utf8]{inputenc}
\usepackage[french]{babel}
\usepackage[T1]{fontenc}
\usepackage{graphicx}
% \usepackage{hyperref}
\usepackage{tikz}
\usepackage[margin=0.75in]{geometry}
\usetikzlibrary{arrows}

\renewcommand{\chaptername}{}
\setcounter{chapter}{-1}

\tikzset{
  treenode/.style = {align=center, inner sep=0pt, text centered,
    font=\sffamily},
  arn_n/.style = {treenode, circle, draw=black, text width=2em},
  arn_x/.style = {treenode, rectangle, draw=black, text width=1.5em,
    minimum width=1.5em, minimum height=1.5em}
}

\title{\Huge Rapport \\ Projet ALGAV Trie \\ \large Implantation du Trie Hybride et du Patricia Trie}
\author{Amel Arkoub 3301571 \\ Ling-Chun SO 3414546}
\date{22 décembre 2017}

\pdfinfo{%
  /Title    ()
  /Author   ()
  /Creator  ()
  /Producer ()
  /Subject  ()
  /Keywords ()
}

\begin{document}
\maketitle

\tableofcontents
\newpage

\chapter{Introduction}
\section*{Présentation}
\paragraph*{}
Nous souhaitons représenter un dictionnaire de mots, c'est-à-dire implanter une structure de
données éfficace stockant des mots. Pour cela, nous allons nous servir de la structure de Trie.
Dans cette optique, nous proposons l'implantation de deux structures de tries concurrentes que sont le Trie Hybride
et le Patricia Trie. De cette implantation, une étude expérimentale sera analysées afin de mettre en exergue
les avantages et inconvénients de chacune de ces structures. Le langage choisit pour l'implantation de ces deux
structures est JAVA.

\section*{Trie}
Un Trie est une représentation arborescente d'un ensemble de clés en évitant de répéter les préfixes communs.

\section*{Trie Hybride}
Un Trie Hybride est un arbre ternaire dont chaque noeud contient un caractère et une valeur (non vide
lorsque le noeud représente une clé). Chaque noeud a 3 pointeurs: un lien Inf (resp. Eq, res. Sup) vers le sous
arbre dont le premier caractère est inférieur (resp. égal, resp. supérieur) à son caractère. Il permet en outre
de réduire le nombre de pointeurs vide par rapport à un R-Trie.

\section*{Patricia Trie}
Un Patricia Trie est un arbre dont le but est de réduire la taille des R-Trie tout en conservant une recherche
efficace. Pour ce faire, plutôt que chaque noeud interne permette de distinguer une lettre, il permet de distinguer
la plus longue sous-chaîne de lettres communes à plusieurs mots.

\chapter{Trie Hybride}
\section{Implantation}
\subsection{Structure}
Dans notre implantation JAVA, un Trie Hybride est un objet qui contient 5 éléments:
\begin{itemize}
 \item un char ``letter'', qui correspond à la lettre stockée.
 \item une valeur ``value'', qui correspond à la représentation de fin de mot (-1 le noeud n'est pas la fin
 d'un mot, sinon la valeur correspond à l'ordre d'ajout croissant).
 \item un pointeur vers un Trie Hybride ``fc'', ce Trie Hybride correpond au fils central, c'est-à-dire on parcours ce
 Trie Hybride si le caractère recherché au Trie Hybride courant est correct.
 \item un pointeur vers un Trie Hybride ``fg'', ce Trie Hybride correpond au fils gauche, c'est-à-dire on parcours ce
 Trie Hybride si le caractère recherché au Trie Hybride courant est plus petit dans l'ordre alphabétique.
 \item un pointeur vers un Trie Hybride ``fd'', ce Trie Hybride correpond au fils droit, c'est-à-dire on parcours ce
 Trie Hybride si le caractère recherché au Trie Hybride courant est plus grand dans l'ordre alphabétique.
\end{itemize}

\paragraph*{}
Voici une représentation ci-dessous du Trie Hybride résultant des l'ajouts successifs des mots:
\begin{itemize}
 \item don
 \item le
 \item a
 \item dort
\end{itemize}

\begin{tikzpicture}[->,>=stealth',level/.style={sibling distance = 6cm/#1,
  level distance = 2.5cm}]
\node [arn_n] {d \\ -1}
    child{ node [arn_n] {a \\ 2}}
    child{ node [arn_n] {o \\ -1}
	    child{ node [arn_x] {}}
            child{ node [arn_n] {n \\ 0} 
							child{ node [arn_n] {r \\ -1}
							      child{ node [arn_n] {t \\ 3}}
							}
							child{ node [arn_x] {}}
							child{ node [arn_x] {}}
            }
            child{ node [arn_x] {}}
		}
    child{ node [arn_n] {l \\ -1}
	    child{ node [arn_n] {e \\ 1}}
    }
; 
\end{tikzpicture}
\begin{figure}[!htbp]
\caption{Représentation d'un Trie Hybride}
\end{figure}

\section{Description des algorithmes}


\subsection{Trie Hybride}



\section{Complexité}
\subsection{Recherche}
Pour la recherche d'un mot de \textit{L} caractères et un Trie Hybride de taille \textit{n}, 
en prenant la comparaison de caractère comme mesure, on obtient une complexité dans le cas général:
\begin{itemize}
 \item en $\Theta$(L), si \textit{L} < \textit{n};
 \item en $\Theta$(n) sinon.
\end{itemize}
Sinon on a une complexité en $\Theta$(n) dans le pire cas.
\subsection{Comptage Mots}
Le comptage de mots revient à faire un parcours de l'arbre en entier.
Soit un arbre de taille \textit{n}, en prenant la comparaison de la valeur du noeud comme mesure,
on a donc une complexité $\Theta$(n).
\subsection{Liste Mots}
La récupération des mots dans un Trie Hybride correspond aussi à un parcours de l'arbre en entier en gardant le \textit{préfixe}
en argument dans les appels de fonctions.
Soit un arbre de taille \textit{n}, en prenant la comparaison de la valeur du noeud comme mesure,
on a donc une complexité en $\Theta$(n).
\subsection{Comptage Nil}
Le comptage de pointeur vers null correspond à un parcours de tout les noeuds pour compter le nombre de fils null.
Soit un arbre de taille \textit{n}, en prenant la comparaison de la valeur du noeud comme mesure,
on a une complexité en $\Theta$(n) puisque pour \textit{n} noeuds, on a un nombre de comparaison $\le$ \textit{3n}.
\subsection{Hauteur}
La détermination de la hauteur du Trie Hybride est un parcours complet de l'arbre, en prenant l'existence de fils comme mesure
, on obtient une complexité en $\Theta$(n).
\subsection{Profondeur Moyenne}
La profondeur moyenne d'un Trie Hybride est un parcours jusqu'aux feuilles dont on ajoute la profondeur dans une liste et on effectue
une division. En prenant la comparaison d'existence de fils comme mesure, on obtient une complexité de $\Theta$(n).
\subsection{Préfixe}
La recherche du préfixe peut dans le pire des cas en $\Theta$(n) puisque le pire des cas est atteint lorsque le Trie Hybride
revient à être une ``liste chaînée'' et donc à hauteur \textit{h}=n.
\subsection{Suppression}
La suppression d'un mot est une recherche dans le Trie Hybride, ce qui correspond à une complexité en $\Theta$(n) comparaison
dans le pire cas, cependant il y a aussi 3 appels à la fonction comptageMots de complexité $\Theta$(n).
On a donc une complexité en $\mathcal{O}$($n^2$).
\chapter{Patricia Trie}
\section{Implantation}
\section{Complexité}
\subsection{Recherche}
\subsection{Comptage Mots}
\subsection{Liste Mots}
\subsection{Comptage Nil}
\subsection{Hauteur}
\subsection{Profondeur Moyenne}
\subsection{Préfixe}
\subsection{Suppression}
\chapter{Fonctions complexes}
\section{Implantation}
\section{Complexité}
\subsection{Fusion de Patricia Trie}
\subsection{Conversion de Patricia Trie en Trie Hybride}
\subsection{Conversion de Trie Hybride en Patricia Trie}
\subsection{Rééquilibrage de Trie Hybride}
\chapter{Etude expérimentale}

\end{document}
