\documentclass[a4paper,12pt]{report}
%\documentclass[a4paper,10pt]{scrartcl}

\usepackage[utf8]{inputenc}
\usepackage[french]{babel}
\usepackage[T1]{fontenc}
\usepackage{graphicx}
\usepackage{hyperref}

\renewcommand{\chaptername}{}

\setcounter{chapter}{-1}

\title{\Huge Rapport \\ Projet ALGAV Trie \\ \large Implantation du Trie Hybride et du Patricia Trie}
\author{Amel Arkoub 3301571 \\ Ling-Chun SO 3414546}
\date{22 décembre 2017}

\pdfinfo{%
  /Title    ()
  /Author   ()
  /Creator  ()
  /Producer ()
  /Subject  ()
  /Keywords ()
}

\begin{document}
\maketitle

\tableofcontents
\newpage
\chapter{Introduction}
\chapter{Trie Hybride}
\section{Implantation}
\section{Complexité}
\subsection{Recherche}
Pour la recherche d'un mot de \textit{L} caractères et un Trie Hybride de taille \textit{n}, 
en prenant la comparaison de caractère comme mesure, on obtient une complexité dans le cas général:
\begin{itemize}
 \item en $\Theta$(L), si \textit{L} < \textit{n};
 \item en $\Theta$(n) sinon.
\end{itemize}
Sinon on a une complexité en $\Theta$(n) dans le pire cas.
\subsection{Comptage Mots}
Le comptage de mots revient à faire un parcours de l'arbre en entier.
Soit un arbre de taille \textit{n}, en prenant la comparaison de la valeur du noeud comme mesure,
on a donc une complexité $\Theta$(n).
\subsection{Liste Mots}
La récupération des mots dans un Trie Hybride correspond aussi à un parcours de l'arbre en entier en gardant le \textit{préfixe}
en argument dans les appels de fonctions.
Soit un arbre de taille \textit{n}, en prenant la comparaison de la valeur du noeud comme mesure,
on a donc une complexité en $\Theta$(n).
\subsection{Comptage Nil}
Le comptage de pointeur vers null correspond à un parcours de tout les noeuds pour compter le nombre de fils null.
Soit un arbre de taille \textit{n}, en prenant la comparaison de la valeur du noeud comme mesure,
on a une complexité en $\Theta$(n) puisque pour \textit{n} noeuds, on a un nombre de comparaison $\le$ \textit{3n}.
\subsection{Hauteur}
La détermination de la hauteur du Trie Hybride est un parcours complet de l'arbre, en prenant l'existence de fils comme mesure
, on obtient une complexité en $\Theta$(n).
\subsection{Profondeur Moyenne}
La profondeur moyenne d'un Trie Hybride est un parcours jusqu'aux feuilles dont on ajoute la profondeur dans une liste et on effectue
une division. En prenant la comparaison d'existence de fils comme mesure, on obtient une complexité de $\Theta$(n).
\subsection{Préfixe}
La recherche du préfixe peut dans le pire des cas en $\Theta$(n) puisque le pire des cas est atteint lorsque le Trie Hybride
revient à être une ``liste chaînée'' et donc à hauteur \textit{h}=n.
\subsection{Suppression}
La suppression d'un mot est une recherche dans le Trie Hybride, ce qui correspond à une complexité en $\Theta$(n) comparaison
dans le pire cas, cependant il y a aussi 3 appels à la fonction comptageMots de complexité $\Theta$(n).
On a donc une complexité en $\mathcal{O}$($n^2$).
\chapter{Patricia Trie}
\section{Implantation}
\section{Complexité}
\subsection{Recherche}
\subsection{Comptage Mots}
\subsection{Liste Mots}
\subsection{Comptage Nil}
\subsection{Hauteur}
\subsection{Profondeur Moyenne}
\subsection{Préfixe}
\subsection{Suppression}
\chapter{Fonctions complexes}
\section{Implantation}
\section{Complexité}
\subsection{Fusion de Patricia Trie}
\subsection{Conversion de Patricia Trie en Trie Hybride}
\subsection{Conversion de Trie Hybride en Patricia Trie}
\subsection{Rééquilibrage de Trie Hybride}
\chapter{Etude expérimentale}

\end{document}
